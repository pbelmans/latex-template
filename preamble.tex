\usepackage{hyperref}
\hypersetup{hypertexnames = false, bookmarksdepth = 2, bookmarksopen = true, colorlinks, linkcolor = black, citecolor = black, urlcolor = black, pdfstartview={XYZ null null 1}}

% mathematics
\usepackage{amsfonts}
\usepackage[fleqn, leqno]{amsmath}
\usepackage{amsthm}

% references
\usepackage[capitalise,noabbrev]{cleveref}

\usepackage{booktabs}
\usepackage{colonequals}
\usepackage{enumitem}
\usepackage{fullpage}
\usepackage{mathtools}
\usepackage{parskip}
\usepackage{thmtools}
\usepackage{tikz-cd}
\usepackage{xparse}
\usepackage{xpatch}
\usepackage{xspace}

\usepackage[utf8]{inputenc}
\usepackage[T1]{fontenc}
\usepackage{libertine}
\usepackage[libertine]{newtxmath}
\usepackage[scaled=0.83]{beramono}
%\usepackage[charter]{mathdesign}
%\usepackage[scaled]{beramono,berasans}
\usepackage{eucal}
\usepackage{microtype}
\frenchspacing

\usepackage{gitinfo2}

% bibliography
\begin{filecontents*}{mrnumber.dbx}
\DeclareDatamodelFields[type=field,datatype=verbatim]{mrnumber}
\DeclareDatamodelEntryfields{mrnumber}
\end{filecontents*}

\usepackage[backend=biber, maxbibnames=99, datamodel=mrnumber, sortcites]{biblatex}

\DeclareFieldFormat{mrnumber}{%
  MR\addcolon\space
  \ifhyperref
    {\href{http://www.ams.org/mathscinet-getitem?mr=#1}{\nolinkurl{#1}}}
    {\nolinkurl{#1}}}

\renewbibmacro*{doi+eprint+url}{%
  \iftoggle{bbx:doi}
    {\printfield{doi}}
    {}%
  \newunit\newblock
  \printfield{mrnumber}%
  \newunit\newblock
  \iftoggle{bbx:eprint}
    {\usebibmacro{eprint}}
    {}%
  \newunit\newblock
  \iftoggle{bbx:url}
    {\usebibmacro{url+urldate}}
    {}}

\DeclareSourcemap{
  \maps[datatype=bibtex]{
    \map{
      \step[fieldset=issn, null]
    }
  }
}
\xpretobibmacro{bib+doi+url}
  {\iffieldundef{doi}
     {}
     {\clearfield{url}}}
  {}{}

% patch macro used in notes
\xpretobibmacro{cite+doi+url}
  {\iffieldundef{doi}
     {}
     {\clearfield{url}}}
  {}{}

% mathematics configuration
\relpenalty=10000
\binoppenalty=10000

% environments
\declaretheoremstyle[spaceabove = 3pt, spacebelow = 3pt, bodyfont = \itshape]{theorem}
\declaretheoremstyle[spaceabove = 3pt, spacebelow = 3pt]{remark}

\declaretheorem[style=theorem]{theorem}
\declaretheorem[style=theorem, sibling=theorem]{corollary}
\declaretheorem[style=theorem, sibling=theorem]{lemma}
\declaretheorem[style=theorem, sibling=theorem]{proposition}

\declaretheorem[style=remark, sibling=theorem]{definition}
\declaretheorem[style=remark, sibling=theorem]{example}
\declaretheorem[style=remark, sibling=theorem]{remark}

\declaretheorem[style=theorem, numberwithin=section, title=Theorem]{alphatheorem}
\declaretheorem[style=theorem, sibling=alphatheorem, title=Conjecture]{alphaconjecture}
\declaretheorem[style=theorem, sibling=alphatheorem, title=Corollary]{alphacorollary}
\declaretheorem[style=theorem, sibling=alphatheorem, title=Proposition]{alphaproposition}

\renewcommand{\thealphatheorem}{\Alph{alphatheorem}}
\renewcommand{\thealphaconjecture}{\Alph{alphatheorem}}
\renewcommand{\thealphacorollary}{\Alph{alphatheorem}}
\renewcommand{\thealphaproposition}{\Alph{alphatheorem}}
\crefname{alphatheorem}{Theorem}{Theorems}
\crefname{alphaconjecture}{Conjecture}{Conjectures}
\crefname{alphacorollary}{Corollary}{Corollaries}
\crefname{alphaproposition}{Proposition}{Propositions}

\crefformat{enumi}{#2\textup{(#1)}#3}

\newcommand\gitfootnote[1]{%
  \begin{NoHyper}
  \renewcommand\thefootnote{}\footnote{#1}%
  \addtocounter{footnote}{-1}%
  \end{NoHyper}
}

\setlist[enumerate]{font=\normalfont}

\newcommand\pieter[1]{\begin{color}{magenta}\textbf{Pieter}: #1\end{color}}
%\newcommand\coauthor[1]{\begin{color}{blue}\textbf{Coauthor}: #1\end{color}}
%\newcommand\coauthor[1]{\begin{color}{red}\textbf{Coauthor}: #1\end{color}}

% macros
\mathchardef\mhyphen="2D
\newcommand\dash{\nobreakdash-\hspace{0pt}}
